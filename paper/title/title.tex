%%
%% The "title" command has an optional parameter,
%% allowing the author to define a "short title" to be used in page headers.
%\title[Thematic Elements]{Using Thematic Elements to Interpret Book Success Prediction}
%\title[Thematic Elements]{Using Semantic Word Associations to Interpret Book Success Prediction}
\title[Semantic Success]{Using Semantic Word Associations to Predict and Interpret the Success of Novels}

% \author{Henry W. Gorelick}
% \email{hgorelick@fordham.edu}
% \orcid{tbd}
% \author{Dr. Mohammad Ruhul Amin}
% \email{mamin17@fordham.edu}
% \affiliation{%
%   \institution{Fordham University}
%   \streetaddress{113 W 60th St}
%   \city{New York}
%   \state{New York}
%   \postcode{10023}
% }

\begin{abstract}
  Literary analysis, in the traditional sense, is the subjective practice of dissecting a work of text to discern deeper meaning.
  Recently however, researchers have taken up the task of adapting literary analysis, conventionally exclusive only to publishers, editors, English professors, and the like, to something data science can recognize.
  And, there has been some success in this venture.
  In this paper, we attempt to predict the success of a novel by modeling the lexical semantic relations of its contents.
  We then analyze those relationships to identify those that directly impact a book's success. 
  We built upon the previous research in this field and created the largest dataset used in such a project containing various types of lexical data from 18,000 books. 
  We implemented the most accurate models to date for predicting book success with domain specific feature reduction techniques, achieving a highest average accuracy of 95.4\%. 
  While such strong performance in success prediction is impressive, we dug deeper to interpret the high accuracy.
  We found a mapping from WordNet's defined semantic word relations to a set of themes as defined in \textit{Roget's Thesaurus}.
  With this mapping, we discovered the themes that successful books of a given genre prioritize.
  In other words, if you want to write a bad children's book, write about keeping quiet in school.
  
\end{abstract}

\begin{CCSXML}
<ccs2012>
   <concept>
       <concept_id>10002951.10003227.10003351</concept_id>
       <concept_desc>Information systems~Data mining</concept_desc>
       <concept_significance>500</concept_significance>
       </concept>
   <concept>
       <concept_id>10002951.10003260.10003277.10003279</concept_id>
       <concept_desc>Information systems~Data extraction and integration</concept_desc>
       <concept_significance>500</concept_significance>
       </concept>
   <concept>
       <concept_id>10002951.10003260.10003261.10003267</concept_id>
       <concept_desc>Information systems~Content ranking</concept_desc>
       <concept_significance>300</concept_significance>
       </concept>
   <concept>
       <concept_id>10010147.10010257.10010293.10010075.10010295</concept_id>
       <concept_desc>Computing methodologies~Support vector machines</concept_desc>
       <concept_significance>500</concept_significance>
       </concept>
   <concept>
       <concept_id>10010147.10010257.10010339</concept_id>
       <concept_desc>Computing methodologies~Cross-validation</concept_desc>
       <concept_significance>300</concept_significance>
       </concept>
   <concept>
       <concept_id>10010147.10010257.10010321.10010336</concept_id>
       <concept_desc>Computing methodologies~Feature selection</concept_desc>
       <concept_significance>500</concept_significance>
       </concept>
   <concept>
       <concept_id>10010147.10010178.10010179.10010184</concept_id>
       <concept_desc>Computing methodologies~Lexical semantics</concept_desc>
       <concept_significance>500</concept_significance>
       </concept>
   <concept>
       <concept_id>10010147.10010178.10010179.10003352</concept_id>
       <concept_desc>Computing methodologies~Information extraction</concept_desc>
       <concept_significance>500</concept_significance>
       </concept>
   <concept>
       <concept_id>10010147.10010178.10010179</concept_id>
       <concept_desc>Computing methodologies~Natural language processing</concept_desc>
       <concept_significance>500</concept_significance>
       </concept>
 </ccs2012>
\end{CCSXML}

\ccsdesc[500]{Information systems~Data mining}
\ccsdesc[500]{Information systems~Data extraction and integration}
\ccsdesc[300]{Information systems~Content ranking}
\ccsdesc[500]{Computing methodologies~Support vector machines}
\ccsdesc[300]{Computing methodologies~Cross-validation}
\ccsdesc[500]{Computing methodologies~Feature selection}
\ccsdesc[500]{Computing methodologies~Lexical semantics}
\ccsdesc[500]{Computing methodologies~Information extraction}
\ccsdesc[500]{Computing methodologies~Natural language processing}

%%
%% Keywords. The author(s) should pick words that accurately describe
%% the work being presented. Separate the keywords with commas.
\keywords{book success prediction, semantic word association, feature reduction, book content mining}

%%
%% This command processes the author and affiliation and title
%% information and builds the first part of the formatted document.
\maketitle