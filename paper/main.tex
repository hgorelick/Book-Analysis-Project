\documentclass[sigconf]{acmart}

%%
%% \BibTeX command to typeset BibTeX logo in the docs
\AtBeginDocument{%
  \providecommand\BibTeX{{%
    \normalfont B\kern-0.5em{\scshape i\kern-0.25em b}\kern-0.8em\TeX}}}
    
\setcopyright{tbd}
\copyrightyear{2020}
\acmYear{2021}
\acmDOI{tbd}

\acmConference[WSDM 2021]{The 14th ACM International Conference on Web
Search and Data Mining}{March 8--12, 2021 }{Jerusalem, Israel}
\acmBooktitle{The 14th ACM International Conference on Web Search and Data Mining, 
March 8--12, 2021, Jerusalem, Israel}
\acmPrice{tbd}
\acmISBN{tbd}

\usepackage{times}
\usepackage{url}
\usepackage{latexsym}
\usepackage{multirow}
% \usepackage[verbose]{wrapfig}
\usepackage{tabto}
\usepackage{enumitem}
\usepackage{hhline}
\usepackage{dblfloatfix}
\usepackage{graphicx}
\usepackage{subcaption}

\graphicspath{ {figures/} }

\usepackage{titlesec}
\titleformat*{\subsection}{\large\bfseries}

\usepackage[utf8]{inputenc}

\usepackage{listings}
\usepackage{algorithm}
% \usepackage{arevmath}
\usepackage[noend]{algpseudocode}
\renewcommand{\algorithmicrequire}{\textbf{Input:}}

\usepackage{xcolor}

% \usepackage[bookmarksopen, bookmarksdepth=2, breaklinks=true]{url}

\definecolor{codegreen}{rgb}{0,0.6,0}
\definecolor{codegray}{rgb}{0.5,0.5,0.5}
\definecolor{codepurple}{rgb}{0.58,0,0.82}
\definecolor{backcolour}{rgb}{0.96,0.96,0.96}

\lstdefinestyle{mystyle}
{
    backgroundcolor=\color{backcolour},   
    commentstyle=\color{codegreen},
    keywordstyle=\color{magenta},
    numberstyle=\tiny\color{codegray},
    stringstyle=\color{codepurple},
    basicstyle=\ttfamily\footnotesize,
    breakatwhitespace=false,         
    breaklines=true,                 
    captionpos=b,                    
    keepspaces=true,                 
    numbers=left,                    
    numbersep=5pt,                  
    showspaces=false,                
    showstringspaces=false,
    showtabs=false,                  
    tabsize=2
}

\lstset{style=mystyle}

\renewcommand{\lstlistingname}{Algorithm}
\renewcommand{\lstlistingnamestyle}{\small}

\TabPositions{0.05\linewidth}
\newcommand{\squeezeup}{\vspace{-2.5mm}}

%%
%% Submission ID.
%% Use this when submitting an article to a sponsored event. You'll
%% receive a unique submission ID from the organizers
%% of the event, and this ID should be used as the parameter to this command.
%%\acmSubmissionID{123-A56-BU3}

%%
%% The majority of ACM publications use numbered citations and
%% references.  The command \citestyle{authoryear} switches to the
%% "author year" style.
%%
%% If you are preparing content for an event
%% sponsored by ACM SIGGRAPH, you must use the "author year" style of
%% citations and references.
%% Uncommenting
%% the next command will enable that style.
%%\citestyle{acmauthoryear}


%%
%% end of the preamble, start of the body of the document source.
\begin{document}

%%
%% The "title" command has an optional parameter,
%% allowing the author to define a "short title" to be used in page headers.
%\title[Thematic Elements]{Using Thematic Elements to Interpret Book Success Prediction}
%\title[Thematic Elements]{Using Semantic Word Associations to Interpret Book Success Prediction}
\title[Semantic Success]{Using Semantic Word Associations to Predict and Interpret the Success of Novels}

% \author{Henry W. Gorelick}
% \email{hgorelick@fordham.edu}
% \orcid{tbd}
% \author{Dr. Mohammad Ruhul Amin}
% \email{mamin17@fordham.edu}
% \affiliation{%
%   \institution{Fordham University}
%   \streetaddress{113 W 60th St}
%   \city{New York}
%   \state{New York}
%   \postcode{10023}
% }

\begin{abstract}
  Literary analysis, in the traditional sense, is the subjective practice of dissecting a work of text to discern deeper meaning.
  Recently however, researchers have taken up the task of adapting literary analysis, conventionally exclusive only to publishers, editors, English professors, and the like, to something data science can recognize.
  And, there has been some success in this venture.
  In this paper, we attempt to predict the success of a novel by modeling the lexical semantic relations of its contents.
  We then analyze those relationships to identify those that directly impact a book's success. 
  We built upon the previous research in this field and created the largest dataset used in such a project containing various types of lexical data from 18,000 books. 
  We implemented the most accurate models to date for predicting book success with domain specific feature reduction techniques, achieving a highest average accuracy of 95.4\%. 
  While such strong performance in success prediction is impressive, we dug deeper to interpret the high accuracy.
  We found a mapping from WordNet's defined semantic word relations to a set of themes as defined in \textit{Roget's Thesaurus}.
  With this mapping, we discovered the themes that successful books of a given genre prioritize.
  In other words, if you want to write a bad children's book, write about keeping quiet in school.
  
\end{abstract}

\begin{CCSXML}
<ccs2012>
   <concept>
       <concept_id>10002951.10003227.10003351</concept_id>
       <concept_desc>Information systems~Data mining</concept_desc>
       <concept_significance>500</concept_significance>
       </concept>
   <concept>
       <concept_id>10002951.10003260.10003277.10003279</concept_id>
       <concept_desc>Information systems~Data extraction and integration</concept_desc>
       <concept_significance>500</concept_significance>
       </concept>
   <concept>
       <concept_id>10002951.10003260.10003261.10003267</concept_id>
       <concept_desc>Information systems~Content ranking</concept_desc>
       <concept_significance>300</concept_significance>
       </concept>
   <concept>
       <concept_id>10010147.10010257.10010293.10010075.10010295</concept_id>
       <concept_desc>Computing methodologies~Support vector machines</concept_desc>
       <concept_significance>500</concept_significance>
       </concept>
   <concept>
       <concept_id>10010147.10010257.10010339</concept_id>
       <concept_desc>Computing methodologies~Cross-validation</concept_desc>
       <concept_significance>300</concept_significance>
       </concept>
   <concept>
       <concept_id>10010147.10010257.10010321.10010336</concept_id>
       <concept_desc>Computing methodologies~Feature selection</concept_desc>
       <concept_significance>500</concept_significance>
       </concept>
   <concept>
       <concept_id>10010147.10010178.10010179.10010184</concept_id>
       <concept_desc>Computing methodologies~Lexical semantics</concept_desc>
       <concept_significance>500</concept_significance>
       </concept>
   <concept>
       <concept_id>10010147.10010178.10010179.10003352</concept_id>
       <concept_desc>Computing methodologies~Information extraction</concept_desc>
       <concept_significance>500</concept_significance>
       </concept>
   <concept>
       <concept_id>10010147.10010178.10010179</concept_id>
       <concept_desc>Computing methodologies~Natural language processing</concept_desc>
       <concept_significance>500</concept_significance>
       </concept>
 </ccs2012>
\end{CCSXML}

\ccsdesc[500]{Information systems~Data mining}
\ccsdesc[500]{Information systems~Data extraction and integration}
\ccsdesc[300]{Information systems~Content ranking}
\ccsdesc[500]{Computing methodologies~Support vector machines}
\ccsdesc[300]{Computing methodologies~Cross-validation}
\ccsdesc[500]{Computing methodologies~Feature selection}
\ccsdesc[500]{Computing methodologies~Lexical semantics}
\ccsdesc[500]{Computing methodologies~Information extraction}
\ccsdesc[500]{Computing methodologies~Natural language processing}

%%
%% Keywords. The author(s) should pick words that accurately describe
%% the work being presented. Separate the keywords with commas.
\keywords{book success prediction, semantic word association, feature reduction, book content mining}

%%
%% This command processes the author and affiliation and title
%% information and builds the first part of the formatted document.
\maketitle

\begin{table}[!t]
    \caption{\# of novels per genre and download count thresholds for unsuccessful ($\leq\upsilon^-$) and successful ($\geq\upsilon^+$) classes for the WordNet model}
    \label{tab:thresholds}
    \begin{tabular}{lrrr}
        % \hline
        \centering
        \textsc{Genre} & \textsc{\# Books} & $\upsilon^-$ & $\upsilon^+$ \\
        \hline
        Adventure & 917 & 11 & 143 \\
        Children & 3278 & 7 & 160 \\
        Detective & 285 & 34 & 85 \\
        Drama & 785 & 12 & 265 \\
        Fantasy & 382 & 45 & 163 \\
        Fiction & 5369 & 11 & 72 \\
        Historical Fiction & 961 & 10 & 156 \\
        Humor & 1024 & 9 & 58 \\
        Poetry & 1664 & 11 & 128 \\
        Romance Fiction & 634 & 15 & 144 \\
        Science Fiction & 1748 & 19 & 165 \\
        Short Stories & 915 & 15 & 105 \\
        All & 17,962 & 17 & 79 \\
        \hline
    \end{tabular}
\end{table}

% % \vspace*{-\baselineskip}
% \vspace*{-\baselineskip}
\begin{figure*}[!thb]
    \caption{Roget book success prediction accuracy by genre vs. distance between upper bound and lower bound of download margin}
    \label{fig:roget threshold search by genre}
    \centering
    \includegraphics[width=\textwidth]{figures/pngs/roget threshold search by genre.png}
\end{figure*}
% \vspace*{-\baselineskip}

% \vspace*{-\baselineskip}
% \vspace*{-\baselineskip}
\begin{figure*}[!thb]
    \caption{WordNet book success prediction accuracy by genre vs. difference of $\upsilon^-$ and $\upsilon^+$}
    \label{fig:wn threshold search by genre}
    \centering
    \includegraphics[width=\textwidth]{figures/pngs/wn threshold search by genre.png}
\end{figure*}
% \vspace*{-\baselineskip}

\begin{figure*}[thb]
    \centering
    \caption{Feature reduction process: WordNet success prediction accuracy vs. number of features}
    \label{fig:wn feature reduction}
% \begin{subfigure}{\columnwidth}
    \includegraphics[width=\textwidth]{figures/pngs/wn_feature_reduction_by_genre.png}
    % \par \bigskip
    % \includegraphics[width=\columnwidth]{figures/pngs/wn_feature_reduction_ALL.png}
% \end{subfigure}
\end{figure*}


\section{Introduction}\label{sec:introduction}

Predicting the success of a novel by analyzing its content is a challenging research problem.
Thousands of new books are published every year, and only a fraction of them achieve wide popularity.
So the prediction of a book success could be exceptionally useful to the publishing industry and enable editors to make better decisions. 
Many factors contribute to a book's success including, but not limited to,
plot, setting, character development, etc.
Additionally, there are some other factors that contribute to a book's popularity that an author and publisher cannot control like the time when the book is published, the author's reputation, and the marketing strategy. 
In this paper, we only focus on the content of the book to predict its popularity. 

%Literary analysis is the practice of dissecting a work of text's to discern deeper meaning, and is therefore ultimately subjective. 
%Readers, publishers, editors, etc. cannot use literary analysis to make empirical conclusions about any work of writing.
%The authors of~\cite{ashok2013} were the first to use statistical stylometry to predict the success of a novel based only on the contents of its first 1,000 sentences, and their work showed very promising results with the best model reaching 84\% accuracy when predicting the success of \textsc{Adventure} books. 

\subsection*{Previous Work}
The authors of~\cite{ashok2013} were the first to use statistical stylometry to predict the success of a novel based only on the contents of its first 1,000 sentences. 
Ashok et al. used stylistic approaches, such as uni-gram, bi-gram, distribution of the parts-of-speech, grammatical rules, constituent tags, and sentiment and connotation values as features with a Linear SVM~\cite{LIB} for the classification task. 
The authors used books from 8 total genres, and they were able to achieve an average accuracy of 75.7\% for across all genres excluding \textsc{Historical Fiction}. 

In~\cite{maharjan_multitask}, Maharajan et al. used a set of hand crafted features in combination with a recurrent neural network and generated feature representation to predict the likelihood of novel success.
The authors of~\cite{maharjan_multitask} obtained an average accuracy of 73.5\% for across 8 genres.
They also performed several experiments, including using all the features used in~\cite{ashok2013}, sentiment concepts~\cite{Senticnet}, different readability metrics, doc2vec~\cite{Doc2Vec} representation of a book, and unaligned word2vec~\cite{Word2Vec} model of the book. 

In a more recent work~\cite{maharjan_emotion}, Maharajan et al. used the flow of the emotions across books for success prediction and obtained an F1-score of 69\%.
They divided the book into chunks, counted the frequency of emotional associations for each word using the NRC emotion lexicon~\cite{NRC}, and then employed a recurrent neural network with an attention mechanism to predict both the genre and the success.

\subsection*{New Work and Improvements}
We discovered various issues with the dataset used in~\cite{ashok2013}\footnote{\url{https://www3.cs.stonybrook.edu/~songfeng/success/}} including, but not limited to its size, contents, and uniformity.
This original dataset is quite small as it only includes the first 1,000 sentences from 800 books split into 8 different genres, which are further split into successful and unsuccessful classes, each having 50 books.
Additionally, many of the files included have less than 1,000 sentences, or contain automatically generated text from Project Gutenberg instead of the text from the proper novel.
Finally, the books included are prelabled with their successful/unsuccessful class, which limits further testing.

Considering these issues, we decided to build upon~\cite{ashok2013}, but made following critical changes:
\begin{itemize}
    \item Built the largest dataset containing a total of 17,962 books.
    %\item Implemented our own preprocessing methods.
    \item Analyzed the \textit{entire} content of each book and employed alternative prediction models.
    \item Introduced our feature reduction method to further improve model performance.
\end{itemize}
This leads to the motivation for this research and subsequent hypothesis.
We believed that we could greatly improve upon the results of~\cite{ashok2013} with a cleaner and more complete dataset.
We hypothesized that there must be at least one model that is both more accurate and more general than uni-gram, and from such a model, we could discover more interesting and revealing qualities that separate successful from non-successful books.
Ultimately, through our improved methodology and larger dataset, our best models achieve over 95\% accuracy for success prediction and identify the thematic elements prioritized by successful novels of a given genre.

\section{Dataset Construction}\label{sec:dataset-construction}

\subsection*{Linguistic Models}\label{subsec:models}
We utilized six linguistic models for our quantitative analysis.
Two of the models are our own implementation of models used in~\cite{ashok2013}.
Our four additional models have not been used to make these types of qualitative conclusions until now. These models include WordNet~\cite{bird_klein_loper_2009}, \textit{Roget's Thesaurus}~\cite{roget}, and two other models that map WordNet to different levels of \textit{Roget's Thesaurus}.

\renewcommand{\labelenumi}{\bfseries\Roman{enumi}}
\begin{enumerate}[label=\Roman*,ref=\textbf{\thesection}]
    \item \textbf{Lexical Choices:\enspace}The words used in written documents is frequently employed for various applications, with the most popular lexical model being the n-gram model.
    For our analysis, we utilized the following lexical choice analysis models:
    \begin{itemize}
        \item \textbf{Unigram:\enspace}The frequency of unique words in the text.
        \item \textbf{WordNet:\enspace}WordNet is large lexical database of English words. The WordNet database groups nouns, verbs, adjectives, and adverbs into sets of cognitive synonyms called Synsets. Each Sysnet expresses a distinct concept and is represented by a single word. Since Sysnets represent conceptual synonyms, they are able to be linked through conceptual and semantic relationships~\cite{wordnet}.
        WordNet has a total of 117,659 Synsets, each represented by a single, unique word, and our model uses the frequencies of these Synsets in each book.
        Not only does WordNet fit our semantic relation analysis methodology, but it has been used for the relevant task of metaphor identification in~\cite{mao2018word}.
        \item \textbf{\textit{Roget's Thesaurus}:\enspace}A tree structured thesaurus with six root nodes, which we will refer to as Roget Classes or Classes for short.
        Each Class is divided in sections, which results in 23 total sections.
        These sections represent 23 unique concepts that are both general enough to encompass a wide range of ideas, but also specific enough to retain clear meaning.
        Therefore, we refer to these sections as Themes and they are the critical piece to interpreting the results of class prediction.
        Themes are further divided into subsections, levels, etc. before terminating in 1,039 groups of synonyms, which we will refer to as Categories. 
        The Categories are comprised of 56,769 total words, with about half appearing in multiple Categories~\cite{roget}. 
        Our Roget model uses the frequencies of these Categories in each book.
        Furthermore, the authors of~\cite{aman2008using} demonstrated the possible applications of \textit{Roget's Thesaurus} for emotion detection with natural language processing, and~\cite{roget-summary} used the thesaurus for the related process of text summarizing.
        \item\label{it:wn to roget} \textbf{Mapping WordNet to Roget:\enspace}Since \textit{Roget's Thesaurus} has fewer synonym groups than WordNet (1,039 vs. 117,659), and those groups are hierarchically abstracted with each of the 1,039 Roget Categories belonging to one of the 23 Roget Themes, we mapped WordNet's Synsets to \textit{Roget's Thesaurus} to discover more meaningful insights into the distinct characteristics of successful novels.
        We mapped WordNet to Roget Categories (WNRC), and then subsequently to Roget Themes (WNRT).
    \end{itemize}
    \item \textbf{Part-of-Speech Distribution:\enspace}The authors of~\cite{ashok2013} demonstrated the value of POS tag distribution in success prediction, and~\cite{koppel2006} presented the relationship between POS tagging and genre detection and authorship attribution.
    Therefore, we reevaluated the application of POS tag distribution for success prediction.
    % \item \textbf{Context Free Grammar Rule Distribution:} \quad We also reevaluate the analysis of CFG rule distribution as
    % presented in~\cite{ashok2013}, and use the same four categories:
    % \begin{itemize}
    %     \item $\Gamma$\tab lexical production rules (productions where the right-hand symbol is a terminal symbol).
    %     \item $\Gamma^G$\tab lexical production rules prepended with the grandparent node.
    %     \item $\gamma$\tab nonlexical production rules (productions where the right-hand symbol is a non-terminal
        
    %     \tab symbol).
    %     \item $\gamma^G$\tab nonlexical production rules prepended with the grandparent node.
    % \end{itemize}
\end{enumerate}

\subsection*{Implementation}\label{subsec:implementation}
We used the sci-kit learn implementation of LibLinear SVM with 5-fold cross validation for class prediction~\cite{scikit-learn,LIB}.
Part-of-speech tag features are scaled with unit normalization, while all other features are scaled using tf-idf. We used two strategies for all class prediction tasks: 
\begin{itemize}
    \item predicting class by genre, and
    \item predicting class independent of genre.
\end{itemize}

After the initial training and testing of each model, we employed an exhaustive feature reduction method, similar to our success labeling process, to maximize performance.
For a given model, we start with the mean feature weight learned during training.
We remove all features from the dataset with weights less than the mean feature weight.
Next, we train and test the model on this reduced feature set and record the accuracy.
For each subsequent test, starting at a step value of 0.25, we take only the features with weights greater than or equal to $Mean(Original Weights) + (StdDev(Original Weights) * Step)$.
This process continues, increasing the step value by 0.25 after each iteration, until one of the following conditions is met:
\begin{itemize}
    \item perfect accuracy is achieved,
    \item maximum accuracy is found (determined if consecutive subsequent feature sets produce decreasing performance), or 
    \item the number of features is reduced to less than 1\% of the original number of features.
\end{itemize}
% This processes improved the performance of all except one model tested.

\begin{table*}[!t]
    \caption{Accuracy (\%) of classification results by genre, with/without feature reduction (R) (\textit{best performance in bold})}
    \label{tab:results by genre}
    \begin{tabular}{c|cccccccccccc|c}
        \hline
        \centering
        \multirow{2}{*}{\textsc{Model}} & \multicolumn{12}{|c|}{\textsc{Genre}} & \multirow{2}{*}{\textsc{Avg}} \\
        \cline{2-13}
        & Adv. & Child. & Detective & Drama & Fantasy & Fiction & Hist. Fict. & Humor & Poetry & Romance & Sci-Fi & Short & \\
        \hline
        Unigram & 77.0 & 77.2 & 84.2 & 74.7 & 78.4 & 70.2 & 76.8 & 84.2 & 74.0 & 73.4 & 76.0 & 69.6 & 76.3 \\
        Unigram\textsuperscript{R} & 79.1 & 82.7 & 86.7 & 78.3 & 80.4 & 75.6 & 80.4 & 88.3 & 79.0 & 77.9 & 81.3 & 78.2 & 80.6 \\
        % \hline
        % Bigram & & & & & & & & & & & & & \\
        % Bigram\textsuperscript{R} & & & & & & & & & & & & & \\
        \hline
        POS & 79.7 & 76.4 & 77.2 & 69.1 & 75.4 & 73.7 & 84.9 & 86.7 & 77.6 & 72.9 & 74.9 & 79.3 & 77.3 \\
        POS\textsuperscript{R} & 80.2 & 77.9 & 79.3 & 70.0 & 77.4 & 74.3 & 84.9 & 87.3 & 77.6 & 75.6 & 75.5 & 80.8 & 78.4 \\
        \hline
        Roget & 80.9 & 82.5 & 81.3 & 80.4 & 79.4 & 76.6 & 85.9 & 89.2 & 80.8 & 76.1 & 77.8 & 83.8 & 81.2 \\
        Roget\textsuperscript{R} & 87.2 & 88.5 & 94.9 & 85.9 & 92.0 & 79.9 & 87.8 & 91.5 & 84.5 & 85.2 & 82.1 & 90.5 & 87.5 \\
        \hline
        WordNet & 87.9 & 81.0 & 86.3 & 81.8 & 81.7 & 76.4 & 82.5 & 87.5 & 80.4 & 78.7 & 76.0 & 82.7 & 81.9 \\
        WordNet\textsuperscript{R} & \textbf{98.5} & 92.9 & \textbf{99.5} & 97.0 & 96.1 & \textbf{92.1} & 97.5 & 96.3 & \textbf{90.8} & 97.3 & 91.3 & \textbf{95.4} & \textbf{95.4} \\
        \hline
        WNRC & 94.3 & 89.8 & 92.4 & 92.8 & 86.0 & 82.1 & 97.0 & 94.4 & 82.4 & 91.9 & 84.9 & 86.4 & 89.5 \\
        WNRC\textsuperscript{R} & \textbf{98.5} & \textbf{93.7} & \textbf{99.5} & \textbf{98.3} & \textbf{96.9} & 85.9 & \textbf{100.0} & \textbf{97.3} & 87.5 & \textbf{97.7} & \textbf{93.9} & \textbf{95.4} & \textbf{95.4} \\
        \hline
        WNRT & 80.7 & 77.7 & 82.5 & 82.3 & 73.8 & 74.0 & 93.0 & 87.7 & 77.6 & 85.5 & 78.6 & 84.5 & 81.5 \\
        WNRT\textsuperscript{R} & 90.8 & 78.3 & 87.7 & 84.0 & 79.5 & 74.8 & 92.0 & 88.6 & 79.3 & 83.7 & 81.5 & 84.3 & 83.7 \\
        % \hline
        % \hline
        % \textsc{Average} & 84.4 & 80.6 & 84.6 & 79.4 & 80.0 & 75.5 & 86.5 & 88.3 & 78.9 & 78.6 & 78.1 & 81.1 & 81.3 \\
        % \textsc{Average\textsuperscript{R}} & 89.1 & 85.7 & 91.3 & 85.6 & 87.0 & 80.4 & 90.4 & 91.6 & 83.1 & 86.2 & 84.3 & 87.4 & 86.8 \\
        % \hline
        % $\Gamma$ & & & & & & & & & & & & & \\
        % $\Gamma^R$ & & & & & & & & & & & & & \\
        % \hline
        % $\Gamma^G$ & & & & & & & & & & & & & \\
        % $\Gamma^{G^R}$ & & & & & & & & & & & & & \\
        % \hline
        % $\gamma$ & & & & & & & & & & & & & \\
        % $\gamma^R$ & & & & & & & & & & & & & \\
        % \hline
        % $\gamma^G$ & & & & & & & & & & & & & \\
        % $\gamma^{G^R}$ & & & & & & & & & & & & & \\
        \hline
    \end{tabular}
\end{table*}

\section{Methodology}\label{sec:methodology}

\subsection*{Linguistic Models}\label{subsec:models}
We utilized six linguistic models for our quantitative analysis.
Two of the models are our own implementation of models used in~\cite{ashok2013}.
Our four additional models have not been used to make these types of qualitative conclusions until now. These models include WordNet~\cite{bird_klein_loper_2009}, \textit{Roget's Thesaurus}~\cite{roget}, and two other models that map WordNet to different levels of \textit{Roget's Thesaurus}.

\renewcommand{\labelenumi}{\bfseries\Roman{enumi}}
\begin{enumerate}[label=\Roman*,ref=\textbf{\thesection}]
    \item \textbf{Lexical Choices:\enspace}The words used in written documents is frequently employed for various applications, with the most popular lexical model being the n-gram model.
    For our analysis, we utilized the following lexical choice analysis models:
    \begin{itemize}
        \item \textbf{Unigram:\enspace}The frequency of unique words in the text.
        \item \textbf{WordNet:\enspace}WordNet is large lexical database of English words. The WordNet database groups nouns, verbs, adjectives, and adverbs into sets of cognitive synonyms called Synsets. Each Sysnet expresses a distinct concept and is represented by a single word. Since Sysnets represent conceptual synonyms, they are able to be linked through conceptual and semantic relationships~\cite{wordnet}.
        WordNet has a total of 117,659 Synsets, each represented by a single, unique word, and our model uses the frequencies of these Synsets in each book.
        Not only does WordNet fit our semantic relation analysis methodology, but it has been used for the relevant task of metaphor identification in~\cite{mao2018word}.
        \item \textbf{\textit{Roget's Thesaurus}:\enspace}A tree structured thesaurus with six root nodes, which we will refer to as Roget Classes or Classes for short.
        Each Class is divided in sections, which results in 23 total sections.
        These sections represent 23 unique concepts that are both general enough to encompass a wide range of ideas, but also specific enough to retain clear meaning.
        Therefore, we refer to these sections as Themes and they are the critical piece to interpreting the results of class prediction.
        Themes are further divided into subsections, levels, etc. before terminating in 1,039 groups of synonyms, which we will refer to as Categories. 
        The Categories are comprised of 56,769 total words, with about half appearing in multiple Categories~\cite{roget}. 
        Our Roget model uses the frequencies of these Categories in each book.
        Furthermore, the authors of~\cite{aman2008using} demonstrated the possible applications of \textit{Roget's Thesaurus} for emotion detection with natural language processing, and~\cite{roget-summary} used the thesaurus for the related process of text summarizing.
        \item\label{it:wn to roget} \textbf{Mapping WordNet to Roget:\enspace}Since \textit{Roget's Thesaurus} has fewer synonym groups than WordNet (1,039 vs. 117,659), and those groups are hierarchically abstracted with each of the 1,039 Roget Categories belonging to one of the 23 Roget Themes, we mapped WordNet's Synsets to \textit{Roget's Thesaurus} to discover more meaningful insights into the distinct characteristics of successful novels.
        We mapped WordNet to Roget Categories (WNRC), and then subsequently to Roget Themes (WNRT).
    \end{itemize}
    \item \textbf{Part-of-Speech Distribution:\enspace}The authors of~\cite{ashok2013} demonstrated the value of POS tag distribution in success prediction, and~\cite{koppel2006} presented the relationship between POS tagging and genre detection and authorship attribution.
    Therefore, we reevaluated the application of POS tag distribution for success prediction.
    % \item \textbf{Context Free Grammar Rule Distribution:} \quad We also reevaluate the analysis of CFG rule distribution as
    % presented in~\cite{ashok2013}, and use the same four categories:
    % \begin{itemize}
    %     \item $\Gamma$\tab lexical production rules (productions where the right-hand symbol is a terminal symbol).
    %     \item $\Gamma^G$\tab lexical production rules prepended with the grandparent node.
    %     \item $\gamma$\tab nonlexical production rules (productions where the right-hand symbol is a non-terminal
        
    %     \tab symbol).
    %     \item $\gamma^G$\tab nonlexical production rules prepended with the grandparent node.
    % \end{itemize}
\end{enumerate}

\subsection*{Implementation}\label{subsec:implementation}
We used the sci-kit learn implementation of LibLinear SVM with 5-fold cross validation for class prediction~\cite{scikit-learn,LIB}.
Part-of-speech tag features are scaled with unit normalization, while all other features are scaled using tf-idf. We used two strategies for all class prediction tasks: 
\begin{itemize}
    \item predicting class by genre, and
    \item predicting class independent of genre.
\end{itemize}

After the initial training and testing of each model, we employed an exhaustive feature reduction method, similar to our success labeling process, to maximize performance.
For a given model, we start with the mean feature weight learned during training.
We remove all features from the dataset with weights less than the mean feature weight.
Next, we train and test the model on this reduced feature set and record the accuracy.
For each subsequent test, starting at a step value of 0.25, we take only the features with weights greater than or equal to $Mean(Original Weights) + (StdDev(Original Weights) * Step)$.
This process continues, increasing the step value by 0.25 after each iteration, until one of the following conditions is met:
\begin{itemize}
    \item perfect accuracy is achieved,
    \item maximum accuracy is found (determined if consecutive subsequent feature sets produce decreasing performance), or 
    \item the number of features is reduced to less than 1\% of the original number of features.
\end{itemize}
% This processes improved the performance of all except one model tested.

\section{Experimental Results}\label{sec:experimental-results}

\subsection*{Linguistic Models}\label{subsec:models}
We utilized six linguistic models for our quantitative analysis.
Two of the models are our own implementation of models used in~\cite{ashok2013}.
Our four additional models have not been used to make these types of qualitative conclusions until now. These models include WordNet~\cite{bird_klein_loper_2009}, \textit{Roget's Thesaurus}~\cite{roget}, and two other models that map WordNet to different levels of \textit{Roget's Thesaurus}.

\renewcommand{\labelenumi}{\bfseries\Roman{enumi}}
\begin{enumerate}[label=\Roman*,ref=\textbf{\thesection}]
    \item \textbf{Lexical Choices:\enspace}The words used in written documents is frequently employed for various applications, with the most popular lexical model being the n-gram model.
    For our analysis, we utilized the following lexical choice analysis models:
    \begin{itemize}
        \item \textbf{Unigram:\enspace}The frequency of unique words in the text.
        \item \textbf{WordNet:\enspace}WordNet is large lexical database of English words. The WordNet database groups nouns, verbs, adjectives, and adverbs into sets of cognitive synonyms called Synsets. Each Sysnet expresses a distinct concept and is represented by a single word. Since Sysnets represent conceptual synonyms, they are able to be linked through conceptual and semantic relationships~\cite{wordnet}.
        WordNet has a total of 117,659 Synsets, each represented by a single, unique word, and our model uses the frequencies of these Synsets in each book.
        Not only does WordNet fit our semantic relation analysis methodology, but it has been used for the relevant task of metaphor identification in~\cite{mao2018word}.
        \item \textbf{\textit{Roget's Thesaurus}:\enspace}A tree structured thesaurus with six root nodes, which we will refer to as Roget Classes or Classes for short.
        Each Class is divided in sections, which results in 23 total sections.
        These sections represent 23 unique concepts that are both general enough to encompass a wide range of ideas, but also specific enough to retain clear meaning.
        Therefore, we refer to these sections as Themes and they are the critical piece to interpreting the results of class prediction.
        Themes are further divided into subsections, levels, etc. before terminating in 1,039 groups of synonyms, which we will refer to as Categories. 
        The Categories are comprised of 56,769 total words, with about half appearing in multiple Categories~\cite{roget}. 
        Our Roget model uses the frequencies of these Categories in each book.
        Furthermore, the authors of~\cite{aman2008using} demonstrated the possible applications of \textit{Roget's Thesaurus} for emotion detection with natural language processing, and~\cite{roget-summary} used the thesaurus for the related process of text summarizing.
        \item\label{it:wn to roget} \textbf{Mapping WordNet to Roget:\enspace}Since \textit{Roget's Thesaurus} has fewer synonym groups than WordNet (1,039 vs. 117,659), and those groups are hierarchically abstracted with each of the 1,039 Roget Categories belonging to one of the 23 Roget Themes, we mapped WordNet's Synsets to \textit{Roget's Thesaurus} to discover more meaningful insights into the distinct characteristics of successful novels.
        We mapped WordNet to Roget Categories (WNRC), and then subsequently to Roget Themes (WNRT).
    \end{itemize}
    \item \textbf{Part-of-Speech Distribution:\enspace}The authors of~\cite{ashok2013} demonstrated the value of POS tag distribution in success prediction, and~\cite{koppel2006} presented the relationship between POS tagging and genre detection and authorship attribution.
    Therefore, we reevaluated the application of POS tag distribution for success prediction.
    % \item \textbf{Context Free Grammar Rule Distribution:} \quad We also reevaluate the analysis of CFG rule distribution as
    % presented in~\cite{ashok2013}, and use the same four categories:
    % \begin{itemize}
    %     \item $\Gamma$\tab lexical production rules (productions where the right-hand symbol is a terminal symbol).
    %     \item $\Gamma^G$\tab lexical production rules prepended with the grandparent node.
    %     \item $\gamma$\tab nonlexical production rules (productions where the right-hand symbol is a non-terminal
        
    %     \tab symbol).
    %     \item $\gamma^G$\tab nonlexical production rules prepended with the grandparent node.
    % \end{itemize}
\end{enumerate}

\subsection*{Implementation}\label{subsec:implementation}
We used the sci-kit learn implementation of LibLinear SVM with 5-fold cross validation for class prediction~\cite{scikit-learn,LIB}.
Part-of-speech tag features are scaled with unit normalization, while all other features are scaled using tf-idf. We used two strategies for all class prediction tasks: 
\begin{itemize}
    \item predicting class by genre, and
    \item predicting class independent of genre.
\end{itemize}

After the initial training and testing of each model, we employed an exhaustive feature reduction method, similar to our success labeling process, to maximize performance.
For a given model, we start with the mean feature weight learned during training.
We remove all features from the dataset with weights less than the mean feature weight.
Next, we train and test the model on this reduced feature set and record the accuracy.
For each subsequent test, starting at a step value of 0.25, we take only the features with weights greater than or equal to $Mean(Original Weights) + (StdDev(Original Weights) * Step)$.
This process continues, increasing the step value by 0.25 after each iteration, until one of the following conditions is met:
\begin{itemize}
    \item perfect accuracy is achieved,
    \item maximum accuracy is found (determined if consecutive subsequent feature sets produce decreasing performance), or 
    \item the number of features is reduced to less than 1\% of the original number of features.
\end{itemize}
% This processes improved the performance of all except one model tested.

% \begin{table*}[!t]
    \caption{Top 5 most important themes for classifying \textsc{Children} novels and successful/unsuccessful thematic words}
    \label{tab:child themes}
    \begin{tabular}{l|ll}
        \hline
        \centering
        \multirow{2}{*}{\textsc{Theme}} & \multicolumn{2}{c}{\textsc{Words}} \\
        \cline{2-3}
        & \multicolumn{1}{c|}{Successful} & \multicolumn{1}{c}{Unsuccessful} \\
        \hline
        Affections & enthusiastic, lively, tenderness & inactive, sluggish, dull \\
        Communication of Ideas & secret, untruth, language & school, grammar, taciturnity \\
        Formation of Ideas & incredulity, impossibility, curiosity & dissent, sanity, memory \\
        Moral & gluttony, impurity, selfishness & punishment, virtue, duty \\
        Personal & expecting, blemish, hopelessness & aggravation, dejection, dullness \\
        \hline
    \end{tabular}
\end{table*} % moved to results/main for spacing

\section{Future Work}\label{sec:future work}

The discoveries made in our research are just the beginning of what can be done with our dataset.
In addition to the data utilized for this project, we also extracted bi-gram and context-free grammar production features from each book.
In future work, we will continue to explore the impact of these features in addition to semantic word associations on book success.

We also believe that we could achieve better results through the use of a different success surrogate metric.
The scale of Project Gutenberg's catalog does not correspond to the website's popularity.
Therefore, as we continue this work we will include ratings from popular sites such as Goodreads.com, Amazon, etc. to acquire a even more accurate measurement of success~\cite{goodreads}.

\section{Conclusion}\label{sec:conclusion}

\subsection*{Linguistic Models}\label{subsec:models}
We utilized six linguistic models for our quantitative analysis.
Two of the models are our own implementation of models used in~\cite{ashok2013}.
Our four additional models have not been used to make these types of qualitative conclusions until now. These models include WordNet~\cite{bird_klein_loper_2009}, \textit{Roget's Thesaurus}~\cite{roget}, and two other models that map WordNet to different levels of \textit{Roget's Thesaurus}.

\renewcommand{\labelenumi}{\bfseries\Roman{enumi}}
\begin{enumerate}[label=\Roman*,ref=\textbf{\thesection}]
    \item \textbf{Lexical Choices:\enspace}The words used in written documents is frequently employed for various applications, with the most popular lexical model being the n-gram model.
    For our analysis, we utilized the following lexical choice analysis models:
    \begin{itemize}
        \item \textbf{Unigram:\enspace}The frequency of unique words in the text.
        \item \textbf{WordNet:\enspace}WordNet is large lexical database of English words. The WordNet database groups nouns, verbs, adjectives, and adverbs into sets of cognitive synonyms called Synsets. Each Sysnet expresses a distinct concept and is represented by a single word. Since Sysnets represent conceptual synonyms, they are able to be linked through conceptual and semantic relationships~\cite{wordnet}.
        WordNet has a total of 117,659 Synsets, each represented by a single, unique word, and our model uses the frequencies of these Synsets in each book.
        Not only does WordNet fit our semantic relation analysis methodology, but it has been used for the relevant task of metaphor identification in~\cite{mao2018word}.
        \item \textbf{\textit{Roget's Thesaurus}:\enspace}A tree structured thesaurus with six root nodes, which we will refer to as Roget Classes or Classes for short.
        Each Class is divided in sections, which results in 23 total sections.
        These sections represent 23 unique concepts that are both general enough to encompass a wide range of ideas, but also specific enough to retain clear meaning.
        Therefore, we refer to these sections as Themes and they are the critical piece to interpreting the results of class prediction.
        Themes are further divided into subsections, levels, etc. before terminating in 1,039 groups of synonyms, which we will refer to as Categories. 
        The Categories are comprised of 56,769 total words, with about half appearing in multiple Categories~\cite{roget}. 
        Our Roget model uses the frequencies of these Categories in each book.
        Furthermore, the authors of~\cite{aman2008using} demonstrated the possible applications of \textit{Roget's Thesaurus} for emotion detection with natural language processing, and~\cite{roget-summary} used the thesaurus for the related process of text summarizing.
        \item\label{it:wn to roget} \textbf{Mapping WordNet to Roget:\enspace}Since \textit{Roget's Thesaurus} has fewer synonym groups than WordNet (1,039 vs. 117,659), and those groups are hierarchically abstracted with each of the 1,039 Roget Categories belonging to one of the 23 Roget Themes, we mapped WordNet's Synsets to \textit{Roget's Thesaurus} to discover more meaningful insights into the distinct characteristics of successful novels.
        We mapped WordNet to Roget Categories (WNRC), and then subsequently to Roget Themes (WNRT).
    \end{itemize}
    \item \textbf{Part-of-Speech Distribution:\enspace}The authors of~\cite{ashok2013} demonstrated the value of POS tag distribution in success prediction, and~\cite{koppel2006} presented the relationship between POS tagging and genre detection and authorship attribution.
    Therefore, we reevaluated the application of POS tag distribution for success prediction.
    % \item \textbf{Context Free Grammar Rule Distribution:} \quad We also reevaluate the analysis of CFG rule distribution as
    % presented in~\cite{ashok2013}, and use the same four categories:
    % \begin{itemize}
    %     \item $\Gamma$\tab lexical production rules (productions where the right-hand symbol is a terminal symbol).
    %     \item $\Gamma^G$\tab lexical production rules prepended with the grandparent node.
    %     \item $\gamma$\tab nonlexical production rules (productions where the right-hand symbol is a non-terminal
        
    %     \tab symbol).
    %     \item $\gamma^G$\tab nonlexical production rules prepended with the grandparent node.
    % \end{itemize}
\end{enumerate}

\subsection*{Implementation}\label{subsec:implementation}
We used the sci-kit learn implementation of LibLinear SVM with 5-fold cross validation for class prediction~\cite{scikit-learn,LIB}.
Part-of-speech tag features are scaled with unit normalization, while all other features are scaled using tf-idf. We used two strategies for all class prediction tasks: 
\begin{itemize}
    \item predicting class by genre, and
    \item predicting class independent of genre.
\end{itemize}

After the initial training and testing of each model, we employed an exhaustive feature reduction method, similar to our success labeling process, to maximize performance.
For a given model, we start with the mean feature weight learned during training.
We remove all features from the dataset with weights less than the mean feature weight.
Next, we train and test the model on this reduced feature set and record the accuracy.
For each subsequent test, starting at a step value of 0.25, we take only the features with weights greater than or equal to $Mean(Original Weights) + (StdDev(Original Weights) * Step)$.
This process continues, increasing the step value by 0.25 after each iteration, until one of the following conditions is met:
\begin{itemize}
    \item perfect accuracy is achieved,
    \item maximum accuracy is found (determined if consecutive subsequent feature sets produce decreasing performance), or 
    \item the number of features is reduced to less than 1\% of the original number of features.
\end{itemize}
% This processes improved the performance of all except one model tested.

\bibliographystyle{ACM-Reference-Format}
\bibliography{ref}

\end{document}
